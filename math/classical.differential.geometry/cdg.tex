\documentclass{article}

% Language setting
% Replace `english' with e.g. `spanish' to change the document language
\usepackage[greek]{babel}

% Set page size and margins
% Replace `letterpaper' with `a4paper' for UK/EU standard size
\usepackage[letterpaper,top=2cm,bottom=2cm,left=3cm,right=3cm,marginparwidth=1.75cm]{geometry}

% Useful packages
\usepackage{amsfonts}
\usepackage{amsmath}
\usepackage{amssymb}
\usepackage{graphicx}
\usepackage{physics}
\usepackage[colorlinks=true, allcolors=blue]{hyperref}

\title{Κλασσική Διαφορική Γεωμετρία Ι}
\author{Δημήτριος Μποκής}

\begin{document}
\maketitle

\newpage

\section{Μαθήμα}

\subsection*{Εισαγωγή}

Υπενθυμίσεις από Γραμμική Άλγεβρα. Βάσεις, γραμμικές απεικονίσεις, εσωτερικό, εξωτερικό γινόμενο.

\subsubsection*{Ορισμός}

Έστω $V$ διανυσματικός χώρος και απεικόνιση $V \times V \to \mathbb{R}$, με $(\overrightarrow{v_1}, \overrightarrow{v_2}) \mapsto \langle \overrightarrow{v_1}, \overrightarrow{v_2} \rangle$, τότε η απεικόνιση αυτή θα λέγεται \textbf{εσωτερικό γινόμενο} αν και μόνο αν ισχύουν:

\begin{itemize}
    \item $\langle a\overrightarrow{v_1} + b\overrightarrow{v_2}, \overrightarrow{u} \rangle = 
    a\langle \overrightarrow{v_1}, \overrightarrow{u} \rangle + 
    b\langle \overrightarrow{v_2}, \overrightarrow{u} \rangle$
    \item $\langle \overrightarrow{u}, \overrightarrow{v} \rangle = 
    \langle \overrightarrow{v}, \overrightarrow{u} \rangle$
    \item $\langle \overrightarrow{u}, \overrightarrow{u} \rangle \ge 0$
\end{itemize}

\subsubsection*{Ορισμός}

Θα ορίσουμε το \textbf{μέτρο/μήκος} ενός διανύσματος $\overrightarrow{v} \in V$ σε διανυσματικό χώρο $V$ ως $\norm{\overrightarrow{v}} = \sqrt{<\overrightarrow{v}, \overrightarrow{v}>}$ όπου $<\overrightarrow{v}, \overrightarrow{v}>$ εσωτερικό γινόμενο.

\subsubsection*{Ορισμός}

Το \textbf{σύνηθες εσωτερικό γινόμενο} του διανυσματικού χώρου $\mathbb{R}^n$, με την κανονική βάση, για δύο διανύσματα $\overrightarrow{x}, \overrightarrow{y} \in \mathbb{R}^n$, με $\overrightarrow{x} = (x_1, x_2, \dots, x_n)$, και $\overrightarrow{y} = (y_1, y_2, \dots y_n)$, είναι το $<\overrightarrow{x}, \overrightarrow{y}> = \sum_{i = 1}^n x_iy_i$.

\subsubsection*{Παρατήρηση}

Η παράγωγος μιάς συνάρτησης εσωτερικού γινομένου υπολογίζεται να είναι:

$\dv{}{t} <u(t), v(t)> = <u'(t), v(t)> + <u(t), v'(t)>$.

\subsubsection*{Ορισμός}

Θα ορίσουμε \textbf{γωνία} $\theta$ δύο διανυσμάτων $\overrightarrow{x}, \overrightarrow{y} \in V$ ενώς διανυσματικού χώρου $V$ ως, $\cos{\theta} = \dfrac{<\overrightarrow{x}, \overrightarrow{y}>}{\norm{u}\norm{v}}$.

\subsubsection*{Ορισμός}

Στο διανυσματικό χώρο $\mathbb{R}^3$ μπορούμε να ορίσουμε και το \textbf{εξωτερικό γινόμενο}. Έστω $\overrightarrow{u} = (u_1, u_2, u_3)$ και $\overrightarrow{v} = (v_1, v_2, v_3) \in \mathbb{R}^3$, διανύσματα, τότε το εξωτερικό γινόμενο των $\overrightarrow{u}$ και $\overrightarrow{v}$ ορίζεται να είναι το: \\

$\overrightarrow{u} \times \overrightarrow{v} = \begin{vmatrix} \overrightarrow{e_1} & \overrightarrow{e_2} & \overrightarrow{e_3} \\ u_1 & u_2 & u_3 \\ v_1 & v_2 & v_3\end{vmatrix} = (u_2v_3 - v_2u_3)\overrightarrow{e_1} + \dots + (u_1v_2 - v_1u_2)\overrightarrow{e_3}.$

\newpage

\subsection{Άσκηση}

Έστω $u = (1, 2, 3)$, $v = (0, 1, -1)$, $w = (2, 5, 1)$ με $u, v, w \in \mathbb{R}^3$.

\subsubsection*{Ερώτημα}

Δείξτε πως το $B = \{u, v, w\}$ αποτελεί βάση του $\mathbb{R}^3$.

\subsubsection*{Λύση}

Προκειμένου το $B = \{u, v, w\}$ να αποτελεί βάση πρέπει τα $u, v, w$ να είναι γραμμικά ανεξάρτητα μεταξύ τους, αρκεί λοιπόν να θέσω πίνακα $\begin{bmatrix} u & v & w\end{bmatrix}$. Έστω λοιπόν πίνακας $\mathbf{B'} = \begin{bmatrix} u & v & w\end{bmatrix}$. Αν δείξω πως $rank(\mathbf{B'}) = dim(\mathbf{B'}) = 3$ τότε όλες οι στήλες του πίνακα είναι γραμμικά ανεξάρτητες και άρα τα $u, v, w$ είναι γραμμικά ανεξάρτητα, υπολογίζω τη διακρίνουσα, $det(\mathbf{B'}) = -4$. \\

Αποτέλεσμα $det(\mathbf{B'}) \ne 0 \implies rank(\mathbf{B'}) = 3$ άρα $u, v, w$ γραμμικά ανεξάρτητα και άρα $span(u, v, w) = \mathbb{R}^3$ οπότε τα $u, v, w$ αποτελούν βάση του $\mathbb{R}^3$ και άρα $B = \{u, v, w\}$ βάση.

\subsubsection*{Ερώτημα}

Βρείτε τις συντεταγμένες του διανύσματος $x = (3, 7, 4)$ με τη βάση $B = \{u, v, w\}$, όπου $u = (1, 2, 3)$, $v = (0, 1, -1)$, $w = (2, 5, 1)$ με $u, v, w \in \mathbb{R}^3$.

\subsubsection*{Λύση}

Θα γράψω το διάνυσμα $x$ ως $x = (x_1, x_2, x_3)$ όπου $x_1, x_2, x_3 \in \mathbb{R}$ και αφού το $B$ αποτελεί βάση του $\mathbb{R}^3$ υπάρχουν μοναδικά $x_1, x_2, x_3$ τέτοια ώστε $x = x_1 u + x_2 v + x_3 w$ αρκεί να λύσω το σύστημα χρησιμοποιώντας τον πίνακα από την προηγούμενη άσκηση $\mathbf{B'}$. \\

$\mathbf{B'} \begin{bmatrix}x_1 \\ x_2 \\ x_3\end{bmatrix} = \begin{bmatrix}3 \\ 7 \\ 4\end{bmatrix} = \begin{bmatrix} 1 & 0 & 2 \\ 2 & 1 & 5 \\ 3 & -1 & 1\end{bmatrix}\begin{bmatrix}x_1 \\ x_2 \\ x_3\end{bmatrix} = \begin{bmatrix}3 \\ 7 \\ 4\end{bmatrix}$, άρα $x_1 = 1, x_2 = 1, x_3 = 1$.

\subsubsection*{Ερώτημα}

Να προσδιοριστεί ο πίνακας αλλαγής βάσης από την κανονική βάση στη βάση $B$.

\subsubsection*{Λύση}

Ο πίνακας αλλαγής βάσης από την κανονική βάση στη $B$ είναι ο πίνακας $M_{B}$ όπου αν $x$ διάνυσμα ως προς τη κανονική βάση του $\mathbb{R}^3$ τότε το $M_{B}x$ είναι το διάνυσμα $x$ ως προς τη βάση $B$. Παρατηρώ πως αν $x_B$ το διάνυσμα γραμμένο ως προς τη βάση $B$ και $x_I$ το διάνυσμα γραμμένο ώς προς την κανονική βάση τότε $\mathbf{B'}x_B = x_I$ και ξέρω πως $M_{B}x_I = x_B$, άρα $\mathbf{B'}M_{B}x_I = x_I$ και άρα $M_{B}$ = $\mathbf{B'}^{-1}$. Ακόμα πιο απλά μπορώ να πω πως αφού $\mathbf{B'}x_B = x_I \iff \mathbf{B'}^{-1}\mathbf{B'}x_B = \mathbf{B'}^{-1}x_I \iff x_B = \mathbf{B'}^{-1}x_I$ και άρα ο πίνακας $M_{B}$ που ψάχνω είναι ο $\mathbf{B'}^{-1}$, οι πράξεις παραλήπονται λόγο έλλειψης χώρου.

\newpage

\subsection{Άσκηση}

Έστω $T : \mathbb{R}^3 \to \mathbb{R}^3$ ένας γραμμικός μετασχηματισμός με $T(x, y, z) = (x + y, y + z, z + x)$.

\subsubsection*{Ερώτημα}

Να βρεθεί ο πίνακας της T ως προς την κανονική βάση.

\subsubsection*{Λύση}

Έστω $M_T$ ο πίνακας της $T$, πρέπει $T(\overrightarrow{x}) = M_T\overrightarrow{x}$. \\

Έστω λοιπόν $M_T\overrightarrow{x} = T(\overrightarrow{x}) \iff \begin{bmatrix} t_{0, 0} & t_{0, 1} & t_{0, 2} \\ t_{1, 0} & t_{1, 1} & t_{1, 2} \\ t_{2, 0} & t_{2, 1} & t_{2, 2} \end{bmatrix}\begin{bmatrix} x \\ y \\ z \end{bmatrix} = \begin{bmatrix} x + y \\ y + z \\ z + x \end{bmatrix}$ \\

Λύνοντας τις εξισώσεις παίρνω $t_{0, 0}x + t_{0, 1}y + t_{0, 2}z = x + y$, $t_{1, 0}x + t_{1, 1}y + t_{1, 2}z = y + z$ και $t_{2, 0}x + t_{2, 1}y + t_{2, 2}z = y + z$. \\

$M_T = \begin{bmatrix} 1 & 1 & 0 \\ 0 & 1 & 1 \\ 1 & 0 & 1\end{bmatrix}$.

\subsubsection*{Ερώτημα}

Υπολογίστε την ορίζουσα του πίνακα του προηγούμενου ερωτήματος.

\subsubsection*{Λύση}

Ευκολά υπολογίζουμε πως $det(M_T) = \begin{vmatrix}1 & 1 \\ 0 & 1\end{vmatrix} - \begin{vmatrix}0 & 1 \\ 1 & 1\end{vmatrix} = 2$. Άρα ο πίνακας αντριστρέψιμος.

\subsubsection*{Ερώτημα}

Βρείτε $Ker(M_T)$ και $Im(M_T)$.

\subsubsection*{Λύση}

Πυρήνας είναι όλα εκείνα τα διανύσματα που ικανοποιούν τη σχέση $M_T\overrightarrow{x} = \overrightarrow{0}$ από γραμμική άλγεβρα ξέρω πως αφού ο πίνακας αντριστρέψημος οι γραμμές αλλά και οι στήλες του πίνακα είναι γραμμικά ανεξάρτητα διανύσματα, άρα πολύ απλά μπορώ να πω πως το μόνο διάνυσμα που ανοίκει στο $Ker(M_T)$ είναι το μηδενικό, $\overrightarrow{0}$. Αντίστοιχα αφού $rank(M_T) = dim(M_T)$ τότε $Im(M_T) = \mathbb{R}^{rank(M_T)}$.

\newpage

\section{Μαθήμα}

Θεώρημα πεπλεγμένης συνάρτησης. Όρια και παράγωγος συναρτήσεων μιας μεταβλητής με διανυσματικές τιμές. Ιδιότητες διανυσματικών συναρτήσεων. Καμπύλες και παραμετροποίηση. Εφαπτόμενο διάνυσμα καμπύλης. Παράδειγμα στον κύκλο. Αναπαραμετροποίηση καμπύλης και επιτρεπτοί μετασχηματισμοί.

\subsection{Άσκηση}

Έστω $f : \mathbb{R} \to \mathbb{R} ^ 3$ με $f(t) = (t^2, \sin{t}, e^{2t})$.

\subsubsection*{Ερώτημα}

Υπολογίστε το $\lim_{t \to 0} f(t)$.

\subsubsection*{Λύση}

$\forall \epsilon > 0, \exists \delta > 0 : \forall x \in \mathbb{R} : \norm{x_0 - x} < \delta \implies \norm{f(x) - f(0)} < \epsilon$, από Λογισμό ΙΙΙ ξέρω πως μπορώ να διασπάσω το όριο και να υπολογίσω ξεχωριστά: \\

$lim_{t \to 0} x^2 = 0$, $\forall \epsilon > 0, \exists \delta = \sqrt{\epsilon} : \forall \norm{x} < \delta \implies x^2 < \epsilon$. \\

$lim_{t \to 0} sinx = 0$, $\forall \epsilon > 0, \exists \delta = \epsilon : \forall \norm{x} < \delta \implies \sin{x} < \norm{x} < \epsilon$. \\

$\lim_{t \to 0} e^{2t} = 1$, το $e^{2t} - 1$ θετικό κοντά στο $0$, $\forall \epsilon > 0 \exists \delta = \dfrac{1}{2}\log(\epsilon + 1) : \forall \norm{t} < \delta$ έχουμε: \\

Αν $t \geq 0$:

\[
  0 \leq t < \frac{1}{2}\log(\epsilon + 1)
  \implies 2t < \log(\epsilon + 1)
  \implies e^{2t} < \epsilon + 1
  \implies e^{2t} - 1 < \epsilon.
\]

Αν $t < 0$:

\[
  -\delta < t < 0
  \implies -\frac{1}{2}\log(\epsilon + 1) < t < 0
  \implies -\log(\epsilon + 1) < 2t < 0
  \implies \frac{1}{\epsilon + 1} < e^{2t} < 1
  \implies -\frac{\epsilon}{\epsilon + 1} < e^{2t} - 1 < 0.
\]

Επομένως,

\[
  -\epsilon < e^{2t} - 1 < \epsilon.
\]

\subsubsection*{Ερώτημα}

Βρείτε τις $f'(t)$ και $f''(t)$.

\subsubsection*{Λύση}

Προκειμένου να υπολογίσουμε την παράγωγο μιας συνάρτησης $f : \mathbb{R} \to \mathbb{R}^3$ αρκεί να βρούμε την παράγωγο κάθε συνηστώσας ξεχωριστά. \\

$f_1'(x_0) = \lim_{x \to x_0}\dfrac{x^2 - x_0^2}{x - x_0} = \lim_{x \to x_0}\dfrac{(x - x_0)(x + x_0)}x - x_0 = \lim_{x \to x_0}x + x_0 = 2x_0$. \\

Οι υπόλοιπες αφήνονται για τον αναγνώστη καθώς είναι τετρημένες αποδίξεις Λογισμού Ι.

\newpage

\section{Καμπύλες και Παραμετροποίηση}

Μπορώ να πω ότι μια καμπύλη του $\mathbb{R}^n$ είναι ένα σύνολο $M = \{(x_1, \dots, x_n) \in \mathbb{R}^n / f(x_1, \dots, x_n) = c\}$, όπου $f : \mathbb{R}^n \to \mathbb{R}$ λεία συνάρτηση και $c \in \mathbb{R}$, αυτές είναι οι λεγόμενες ισοσταθμικές καμπύλες καθώς το $c$ είναι σταθερό.

Γενικά θα χρησιμοποιούμε λείες συναρτήσεις π.χ. $f : I \to \mathbb{R}^n$, $t \to (f_1(t), \dots, f_n(t))$.

Ένας άλλος τρόπος να σκεφτώ τι είναι μια καμπύλη, είναι να τη δώ σαν τροχιά ενός κινούμενου σημείου $P$, έτσι αν $\overrightarrow{x}(t)$ είναι το διάνυσμα θέσης του σημείου $P$ συναρτήσει του χρόνου $t$, η καμπύλη είναι το σύνολο $M = \{P(t) \in \mathbb{R}^n / \overrightarrow{0P}(t) = \overrightarrow{x}(t)\}$.

\subsubsection{Ορισμός}

Έστω διανυσματική απεικόνιση $\overrightarrow{x}(t) : I \to \mathbb{R}^n$ παραμετροποιημένη καμπύλη, τότε το σύνολο $M = \{P(t) \in \mathbb{R}^n / \overrightarrow{0P}(t) = \overrightarrow{x}(t)\}$ θα ονομάζεται \textbf{ίχνος} της καμπύλης.

\subsubsection{Ορισμός}

Μια παραμετροποιημένη καμπύλη με διανυσματική απεικόνιση θέσης $\overrightarrow{x}(t) : I \to \mathbb{R}^n$ θα λέγεται \textbf{ομαλή} αν η "ταχύτητα" της δεν μηδενίζεται, δηλαδή $\forall t \in I$ ισχύει πως $\overrightarrow{x}'(t) \ne 0$.

\subsubsection{Παράδειγμα}

Η συνάρτηση $\overrightarrow{f} : \mathbb{R} \to \mathbb{R} ^2$, $\overrightarrow{f}(t) = (t^3, t^6)$ \textbf{δεν} είναι ομαλή παραμετροποίηση της καμπύλης.


\subsubsection{Παράδειγμα}

Βασικές παραμετροποιήσεις σύνηθων καμπυλών.

\begin{itemize}
	\item κύκλου ακίνας $r$, $\overrightarrow{x}(t) = (r \cos{t}, r \sin{t})$.
	\item έλικας, $\overrightarrow{x}(t) = (a \cos{t}, a \sin{t}, bt)$.
	\item παραβολής, $\overrightarrow{x}(t) = (t, t^2)$.
	\item ευθεία, $\overrightarrow{x}(t) = p + tv$, $p,v \in \mathbb{R}^n$.
\end{itemize}

\subsection{Εφαπτόμενο Διάνυσμα}

Είπαμε ότι μια καμπύλη μπορώ να τη σκεφτώ σαν τροχιά ενός κινούμενου σημείου, για παράδειγμα η ευθεία $y = x$ έχει παραμετροποίηση $x_1(t) = t, x_2(t) = t$ και $\overrightarrow{x}(t) : \mathbb{R} \to \mathbb{R} ^ 2$, $t \to (x_1(t), x_2(t))$. Για να προσδιορίσω την ταχύτητα και κατεύθυση χρειάζομαι την παράγωγο της $\overrightarrow{x}(t)$ όπου $\overrightarrow{x}'(t) = (x_1'(t), x_2'(t))$.

\subsubsection{Ορισμός}

Το διάνυσμα $\overrightarrow{x}'(t_0) = (x_1'(t_0), x_2'(t_0))$ ονομάζεται \textbf{εφαπτόμενο διάνυσμα} ή \textbf{διάνυσμα ταχύτητας} της καμπύλης στο $P(t_0)$. Ο φορέας αυτού του διανύσματος είναι η \textbf{εφαπτομένη ευθεία} της καμπύλης στο $P(t_0)$

\subsection{Αναπαραμετροποίηση Καμπύλης}

\end{document}
