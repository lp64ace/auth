\documentclass{article}

% Language setting
% Replace `english' with e.g. `spanish' to change the document language
\usepackage[greek]{babel}

% Set page size and margins
% Replace `letterpaper' with `a4paper' for UK/EU standard size
\usepackage[letterpaper,top=2cm,bottom=2cm,left=3cm,right=3cm,marginparwidth=1.75cm]{geometry}

% Useful packages
\usepackage{amsfonts}
\usepackage{amsmath}
\usepackage{amssymb}
\usepackage{graphicx}
\usepackage[colorlinks=true, allcolors=blue]{hyperref}

\title{Κλασσική Διαφορική Γεωμετρία Ι}
\author{Δημήτριος Μποκής}

\begin{document}
\maketitle

\newpage

\section{Μαθήμα}

Υπενθυμίσεις από Γραμμική Άλγεβρα. Βάσεις, γραμμικές απεικονίσεις, εσωτερικό, εξωτερικό, μικτό γινόμενο.

\subsection{Άσκηση}

Έστω $u = (1, 2, 3)$, $v = (0, 1, -1)$, $w = (2, 5, 1)$ με $u, v, w \in \mathbb{R}^3$.

\subsubsection*{Ερώτημα}

Δείξτε πως το $B = \{u, v, w\}$ αποτελεί βάση του $\mathbb{R}^3$.

\subsubsection*{Λύση}

Προκειμένου το $B = \{u, v, w\}$ να αποτελεί βάση πρέπει τα $u, v, w$ να είναι γραμμικά ανεξάρτητα μεταξύ τους, αρκεί λοιπόν να θέσω πίνακα $\begin{bmatrix} u & v & w\end{bmatrix}$. Έστω λοιπόν πίνακας $\mathbf{B'} = \begin{bmatrix} u & v & w\end{bmatrix}$. Αν δείξω πως $rank(\mathbf{B'}) = dim(\mathbf{B'}) = 3$ τότε όλες οι στήλες του πίνακα είναι γραμμικά ανεξάρτητες και άρα τα $u, v, w$ είναι γραμμικά ανεξάρτητα, υπολογίζω τη διακρίνουσα, $det(\mathbf{B'}) = -4$. \\

Αποτέλεσμα $det(\mathbf{B'}) \ne 0 \implies rank(\mathbf{B'}) = 3$ άρα $u, v, w$ γραμμικά ανεξάρτητα και άρα $span(u, v, w) = \mathbb{R}^3$ οπότε τα $u, v, w$ αποτελούν βάση του $\mathbb{R}^3$ και άρα $B = \{u, v, w\}$ βάση.

\subsubsection*{Ερώτημα}

Βρείτε τις συντεταγμένες του διανύσματος $x = (3, 7, 4)$ με τη βάση $B = \{u, v, w\}$, όπου $u = (1, 2, 3)$, $v = (0, 1, -1)$, $w = (2, 5, 1)$ με $u, v, w \in \mathbb{R}^3$.

\subsubsection*{Λύση}

Θα γράψω το διάνυσμα $x$ ως $x = (x_1, x_2, x_3)$ όπου $x_1, x_2, x_3 \in \mathbb{R}$ και αφού το $B$ αποτελεί βάση του $\mathbb{R}^3$ υπάρχουν μοναδικά $x_1, x_2, x_3$ τέτοια ώστε $x = x_1 u + x_2 v + x_3 w$ αρκεί να λύσω το σύστημα χρησιμοποιώντας τον πίνακα από την προηγούμενη άσκηση $\mathbf{B'}$. \\

$\mathbf{B'} \begin{bmatrix}x_1 \\ x_2 \\ x_3\end{bmatrix} = \begin{bmatrix}3 \\ 7 \\ 4\end{bmatrix} = \begin{bmatrix} 1 & 0 & 2 \\ 2 & 1 & 5 \\ 3 & -1 & 1\end{bmatrix}\begin{bmatrix}x_1 \\ x_2 \\ x_3\end{bmatrix} = \begin{bmatrix}3 \\ 7 \\ 4\end{bmatrix}$, άρα $x_1 = 1, x_2 = 1, x_3 = 1$.

\subsubsection*{Ερώτημα}

Να προσδιοριστεί ο πίνακας αλλαγής βάσης από την κανονική βάση στη βάση $B$.

\subsubsection*{Λύση}

Ο πίνακας αλλαγής βάσης από την κανονική βάση στη $B$ είναι ο πίνακας $M_{B}$ όπου αν $x$ διάνυσμα ως προς τη κανονική βάση του $\mathbb{R}^3$ τότε το $M_{B}x$ είναι το διάνυσμα $x$ ως προς τη βάση $B$. Παρατηρώ πως αν $x_B$ το διάνυσμα γραμμένο ως προς τη βάση $B$ και $x_I$ το διάνυσμα γραμμένο ώς προς την κανονική βάση τότε $\mathbf{B'}x_B = x_I$ και ξέρω πως $M_{B}x_I = x_B$, άρα $\mathbf{B'}M_{B}x_I = x_I$ και άρα $M_{B}$ = $\mathbf{B'}^{-1}$. Ακόμα πιο απλά μπορώ να πω πως αφού $\mathbf{B'}x_B = x_I \iff \mathbf{B'}^{-1}\mathbf{B'}x_B = \mathbf{B'}^{-1}x_I \iff x_B = \mathbf{B'}^{-1}x_I$ και άρα ο πίνακας $M_{B}$ που ψάχνω είναι ο $\mathbf{B'}^{-1}$, οι πράξεις παραλήπονται λόγο έλλειψης χώρου.

\newpage

\subsection{Άσκηση}

Έστω $T : \mathbb{R}^3 \to \mathbb{R}^3$ ένας γραμμικός μετασχηματισμός με $T(x, y, z) = (x + y, y + z, z + x)$.

\subsubsection*{Ερώτημα}

Να βρεθεί ο πίνακας της T ως προς την κανονική βάση.

\subsubsection*{Λύση}

Έστω $M_T$ ο πίνακας της $T$, πρέπει $T(\overrightarrow{x}) = M_T\overrightarrow{x}$. \\

Έστω λοιπόν $M_T\overrightarrow{x} = T(\overrightarrow{x}) \iff \begin{bmatrix} t_{0, 0} & t_{0, 1} & t_{0, 2} \\ t_{1, 0} & t_{1, 1} & t_{1, 2} \\ t_{2, 0} & t_{2, 1} & t_{2, 2} \end{bmatrix}\begin{bmatrix} x \\ y \\ z \end{bmatrix} = \begin{bmatrix} x + y \\ y + z \\ z + x \end{bmatrix}$ \\

Λύνοντας τις εξισώσεις παίρνω $t_{0, 0}x + t_{0, 1}y + t_{0, 2}z = x + y$, $t_{1, 0}x + t_{1, 1}y + t_{1, 2}z = y + z$ και $t_{2, 0}x + t_{2, 1}y + t_{2, 2}z = y + z$. \\

$M_T = \begin{bmatrix} 1 & 1 & 0 \\ 0 & 1 & 1 \\ 1 & 0 & 1\end{bmatrix}$.

\subsubsection*{Ερώτημα}

Υπολογίστε την ορίζουσα του πίνακα του προηγούμενου ερωτήματος.

\subsubsection*{Λύση}

Ευκολά υπολογίζουμε πως $det(M_T) = \begin{vmatrix}1 & 1 \\ 0 & 1\end{vmatrix} - \begin{vmatrix}0 & 1 \\ 1 & 1\end{vmatrix} = 2$. Άρα ο πίνακας αντριστρέψιμος.

\subsubsection*{Ερώτημα}

Βρείτε $Ker(M_T)$ και $Im(M_T)$.

\subsubsection*{Λύση}

Πυρίνας είναι όλα εκείνα τα διανύσματα που ικανοποιούν τη σχέση $M_T\overrightarrow{x} = \overrightarrow{0}$ από γραμμική άλγεβρα ξέρω πως αφού ο πίνακας αντριστρέψημος οι γραμμές αλλά και οι στήλες του πίνακα είναι γραμμικά ανεξάρτητα διανύσματα, άρα πολύ απλά μπορώ να πω πως το μόνο διάνυσμα που ανοίκει στο $Ker(M_T)$ είναι το μηδενικό, $\overrightarrow{0}$. Αντίστοιχα αφού $rank(M_T) = dim(M_T)$ τότε $Im(M_T) = \mathbb{R}^{rank(M_T)}$.

\section{Μαθήμα}

Θεώρημα πεπλεγμένης συνάρτησης. Όρια και παράγωγος συναρτήσεων μιας μεταβλητής με διανυσματικές τιμές. Ιδιότητες διανυσματικών συναρτήσεων. Καμπύλες και παραμετροποίηση. Εφαπτόμενο διάνυσμα καμπύλης. Παράδειγμα στον κύκλο. Αναπαραμετροποίηση καμπύλης και επιτρεπτοί μετασχηματισμοί.



\end{document}
