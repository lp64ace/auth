\documentclass{article}

% Language setting
% Replace `english' with e.g. `spanish' to change the document language
\usepackage[greek]{babel}

% Set page size and margins
% Replace `letterpaper' with `a4paper' for UK/EU standard size
\usepackage[letterpaper,top=2cm,bottom=2cm,left=3cm,right=3cm,marginparwidth=1.75cm]{geometry}

% Useful packages
\usepackage{amsfonts}
\usepackage{amsmath}
\usepackage{amssymb}
\usepackage{graphicx}
\usepackage[colorlinks=true, allcolors=blue]{hyperref}

\title{Κλασσική Διαφορική Γεωμετρία Ι}
\author{Δημήτριος Μποκής}

\begin{document}
\maketitle

\section{Μαθήμα}

Υπενθυμίσεις από Γραμμική Άλγεβρα. Βάσεις, γραμμικές απεικονίσεις, εσωτερικό, εξωτερικό, μικτό γινόμενο.

\subsection{Άσκηση}

Έστω $u = (1, 2, 3)$, $v = (0, 1, -1)$, $w = (2, 5, 1)$ με $u, v, w \in \mathbb{R}^3$, δείξτε πως το $B = \{u, v, w\}$ αποτελεί βάση του $\mathbb{R}^3$.

\subsection*{Λύση}

Προκειμένου το $B = \{u, v, w\}$ να αποτελεί βάση πρέπει τα $u, v, w$ να είναι γραμμικά ανεξάρτητα μεταξύ τους, αρκεί λοιπόν να θέσω πίνακα $\begin{bmatrix} u & v & w\end{bmatrix}$.

Έστω λοιπόν πίνακας $\mathbf{B'} = \begin{bmatrix} u & v & w\end{bmatrix} = \begin{bmatrix} 1 & 0 & 2 \\ 2 & 1 & 5 \\ 3 & -1 & 1\end{bmatrix}$.

Αν δείξω πως $rank(\mathbf{B'}) = dim(\mathbf{B'}) = 3$ τότε όλες οι στήλες του πίνακα είναι γραμμικά ανεξάρτητες και άρα τα $u, v, w$ είναι γραμμικά ανεξάρτητα, υπολογίζω τη διακρίνουσα.

$det(\mathbf{B'}) = \begin{vmatrix} 1 & 0 & 2 \\ 2 & 1 & 5 \\ 3 & -1 & 1\end{vmatrix} = 1 \begin{vmatrix}1 & 5 \\ -1 & 1\end{vmatrix} - 0 \begin{vmatrix}2 & 5 \\ 3 & 1\end{vmatrix} + 2 \begin{vmatrix}2 & 1 \\ 3 & -1\end{vmatrix} = 1 (1 - (-5)) + 2(-2 - 3) = 6 - 10 = -4$

Αποτέλεσμα $det(\mathbf{B'}) \ne 0 \implies rank(\mathbf{B'}) = 3$ άρα $u, v, w$ γραμμικά ανεξάρτητα και άρα $span(u, v, w) = \mathbb{R}^3$ οπότε τα $u, v, w$ αποτελούν βάση του $\mathbb{R}^3$ και άρα $B = \{u, v, w\}$ βάση.

\end{document}
